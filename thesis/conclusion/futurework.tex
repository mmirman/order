\section{Future Work}

This thesis presents a new language, and more work can easily be envisioned to provide a greater
framework for proving theorems about it. In general, compilation for a language
where the programs are theorems for a consistent logic allows significant optimization
capability. In Twelf, totality, modes, and worlds allowed predicates to be converted
to programs. In general, running Caledon programs in the current implementation
is slow, as types need to be recorded and searched during runtime. Algorithms
that take advantage of totality checking \citep{altenkirch2010termination}, 
uniqueness checking \citep{anderson2004verifying}, 
worlds checking\citep{anderson2004verifying}, 
mode checking\citep{anderson2004verifying}, 
and universe checking \citep{harper1991type}, 
could be implemented and applied as they were for Twelf and Agda.  It would be useful to have a type system for a logic programming
language which could ensure closed predicates were theorems. 

More work needs to be done to automate type class instancing, as was demonstrated in the section on Linearity.
While implemented, universe checking during unification has yet to be proven entirely
correct.

The future holds further investigation into additional applications of the language, and
it is clear that much more interesting programs can be written with Caledon. While derivatives of one
holed types are possible in Caledon, automatically providing traversals for these
zipper types is an unexplored topic. While I have demonstrated a concise method of
creating concurrency, I look forward to designing libraries for controlling concurrency using the IO primitives.

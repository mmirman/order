\section{Implicit Calculus of Constructions}

\citet{miquel2001implicit} provides a more general system than that seen 
in Hindley-Milner, $ICC$ to allow for implicit arguments.
Here, I  briefly explain the system and some of the relevant theoretical results that have been obtained.
Because maintaining flexibility is important to future extensions of Caledon, 
I present the implicit calculus in terms of Pure Type Systems.

\begin{figure}[H]
\[ 
E ::= V 
 \orr S 
 \orr E\;E 
 \orr \lambda V . E 
 \orr \Pi V : E . E 
 \orr \forall V : E . E 
\]
\caption{Syntax of $ICC$}
\label{icc:syntax}
\end{figure}

Miquel's presentation of $ICC$ uses Curry-style $\lambda$ bindings with types omitted.  
The typing rules for $ICC$ are predominantly the same as those for Pure Type Systems, 
except that I provide an extra rule
for abstraction, application, and formation of implicitly quantification.  The abstraction rule
also must conform to the syntax of the Curry-style $\lambda$ bindings.

\begin{figure}[H]
\[
\infer[\m{abstraction}]
{
\Gamma \vdash_{ICC} (\lambda x . F) : (\Pi x : A . B)
}
{
\Gamma , x : A\vdash_{ICC} F : B
&
\Gamma \vdash_{ICC} (\Pi x : A . B) : s
&
s \in S
}
\]

\[
\infer[\m{gen}]
{
\Gamma \vdash_{ICC} M : (\forall x : T . U)
}
{
\Gamma , x : T\vdash_{ICC} M : U
&
\Gamma \vdash_{ICC} (\forall x : T . U) : s
&
s \in S
&
x \notin FV(M)
}
\]

\[
\infer[\m{inst}]
{
\Gamma \vdash_{ICC} M : U [N/x]
}
{
\Gamma \vdash_{ICC} M : \forall x :T . U
&
\Gamma \vdash_{ICC} N : T
}
\]

\[
\infer[\m{imp-prod}]
{
\Gamma \vdash_{ICC} (\forall x : A . B) : s_3
}
{
\Gamma \vdash_{ICC} A : s_1
&
\Gamma,x:A \vdash_{ICC} B : s_2
&
(s_1,s_2,s_3) \in R
}
\]


\[
\infer[\m{strength}]
{
\Gamma \vdash_{ICC} M : U
}
{
\Gamma , x : T \vdash_{ICC} M : U
&
x \notin FV(M) \cup FV(U)
}
\]

\[
\infer[\m{ext}]
{
\Gamma \vdash_{ICC} M : T
}
{
\Gamma\vdash_{ICC} \lambda x . (M x)  : T 
&
x \notin FV(M)
}
\]
\caption{Typing for $ICC$}
\label{icc:typing}
\end{figure}

In the formulation in \ref{icc:typing}, 
there is no way to control the type of the argument used explicitly.
Similarly, there is no mechanism for this in the syntax shown in \ref{icc:syntax}.
In the implemented version, this is not the case, as a notion of explicit binding has been provided.

In addition, in the formulation, neither the strengthening rule nor 
the rule of extensionality are admissible. These rules are necessary to show subject reduction.


\subsection{Subtyping}

\begin{definition}
\textbf{Subtyping relation: }
$\Gamma \vdash_{ICC} T \leq T' \;\; \equiv \;\; \Gamma, x : T \vdash_{ICC} x : T'$ 
\end{definition}

\begin{lemma}
\textbf{(Subtyping is a preordering)}

\begin{tabular}{lrc}
$
\infer-[\m{sym}]{ 
\Gamma \vdash_{ICC} T \leq T
}{
\Gamma \vdash_{ICC} T : s
}
$
&
$
\infer-[\m{trans}]{ 
\Gamma \vdash_{ICC} T_1 \leq T_3
}{
\Gamma \vdash_{ICC} T_1 \leq T_2
&
\Gamma \vdash_{ICC} T_2 \leq T_3
}
$
&
$
\infer-[\m{sub}]{ 
\Gamma \vdash_{ICC} M : T'
}{
\Gamma \vdash_{ICC} M \leq T
&
\Gamma \vdash_{ICC} T \leq T'
}
$
\end{tabular}

\end{lemma}

\begin{lemma}
Domains of products are contravariant and codomains are covarient:

\begin{tabular}{lrc}
$
\infer[]{ 
\Gamma \vdash_{ICC} \Pi x : T . U \leq \Pi x : T' . U'
}{
\Gamma \vdash_{ICC} T' \leq T 
&
\Gamma,x : T' \vdash_{ICC} U \leq U'
}
$
&
$
\infer[]{ 
\Gamma \vdash_{ICC} \forall x : T . U \leq \forall x : T' . U'
}{
\Gamma \vdash_{ICC} T' \leq T 
&
\Gamma,x : T' \vdash_{ICC} U \leq U'
}
$
\end{tabular}
\end{lemma}

\subsection{Results}

There are two main results that follow from this calculus.

\begin{theorem}
\textbf{(Subject Reduction)} If $\Gamma \vdash_{ICC} M : T$ and $M \rightarrow_{\beta\eta*} M'$ then $\Gamma \vdash_{ICC} M' : T$
\end{theorem}

\begin{definition}
$ \m{Term}_{ICC}  = \{ M : \exists T,\Gamma . \Gamma \vdash_{ICC} M : T \}$
\end{definition}

Because this calculus is Curry-style, Church-Rosser is provable.  
While the internal representation and external presentation of Caledon is not necessarily Curry-style, 
it is possible to mimic a Church-style encoding into a Curry-style encoding through the use of type ascriptions
and evaluation-delaying terms.  Technically, the calculus will no longer have the Church-Rosser property if 
evaluation-delaying terms are included. However, evaluation-delaying terms are ineffectual when added to a strongly normalizing calculus.


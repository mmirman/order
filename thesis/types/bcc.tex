\section{Bicolored Syntax}

The Bicolored Calculus of Constructions \citep{luther2001more}, $CC^{Bi}$
can be considered a varient of the Curry-style CC where there and two products, and two applications as seen in \ref{ccbi:syntax}.
The usefullness of $CC^{Bi}$ comes from the fact that the
consistency of the $CC^{Bi}$ is a direct result of the consistency of CC.
Because of this, one can use $CC^{Bi}$ as a target language for the result of implicit argument 
generation of a system with omitted implicit types, and obtain consistency of 
the underlying monochromatic $\lambda$ calculus from the conversion,
given that reductions are not typed in this calculus.

\begin{figure}[H]
\[ 
E ::= V 
 \orr S 
 \orr E\;E 
 \orr E\;[E]
 \orr \lambda V . E 
 \orr \Pi V : E . E 
 \orr \forall V : E . E 
\]
\caption{Syntax of $CC^{Bi}$}
\label{ccbi:syntax}
\end{figure}

Because it is curry style and contains only one binder, it is not sufficient 
as a target for Caledon's system.  Instead, we work with Barras' Church-style $ICC*$ \citep{barras2008implicit}, 
which has two products, two binders, and two applications as seen in \ref{icc*:syntax}.

\begin{figure}[H]
\[ 
E ::= V 
 \orr S 
 \orr E\;E 
 \orr E\;[E]
 \orr \lambda (V : T). E 
 \orr \lambda [V : T] . E 
 \orr \Pi (V : E) . E 
 \orr \Pi [V : E] . E 
\]
\caption{Syntax of $ICC*$}
\label{icc*:syntax}
\end{figure}

In order to write the typing rules for this calculus, 
the projection in \ref{icc*:proj} from terms of this calculus 
to terms of $ICC$ is necessary.  

\begin{figure}[H]
\[ 
V* := V
\]

\[
(E_1\;E_2)* := E_1* \; E_2*
\]

\[
(E_1\;[E_2])* := E_1*
\]

\[
(\lambda (V : T). E )* := \lambda V. E*
\]

\[
(\lambda [V : T]. E )* := E*
\]

\[
(\Pi [V : T]. E )* := E*
\]

\[
(\Pi (V : T). E )* := \Pi (V : T*). E*
\]
\caption{Projection from $ICC*$ to $ICC$}
\label{icc*:proj}
\end{figure}

The condition in the abstraction* 
rule that variables must not be abstracted from the projected term
is equivalent to the restriction that implicit abstraction is 
non implicit proof irrelevant.

\begin{figure}[H]

\[
\infer[\m{axiom}]
{
\Gamma \vdash_{*} * : \Box
}
{
}
\]

\[
\infer[\m{start}]
{
\Gamma,x:A \vdash_{*} x :A
}
{
\Gamma \vdash_{*} A:K
}
\]


\[
\infer[\m{weakening}]
{
\Gamma,x:C \vdash_{*} A:B
}
{
\Gamma \vdash_{*} A:B
&
\Gamma \vdash_{*} C:K
}
\]


\[
\infer[\m{product}]
{
\Gamma \vdash_{*} (\Pi x : A . B) : K_1
}
{
\Gamma,x:A \vdash_{*} B : K_2
&
\Gamma \vdash_{*} A : K_1
}
\]

\[
\infer[\m{product}*]
{
\Gamma \vdash_{*} (\Pi[x : A]. B) : K_1
}
{
\Gamma,x:A \vdash_{*} B : K_2
&
\Gamma \vdash_{*} A : K_1
}
\]

\[
\infer[\m{app}]
{
\Gamma \vdash_{*} F V : [V/x] B
}
{
\Gamma \vdash_{*} F : (\Pi x : A . B)
&
\Gamma \vdash_{*} V : A
&
\text{x is free for V in B}
}
\]

\[
\infer[\m{app*}]
{
\Gamma \vdash_{*} F [V] : [V/x] B
}
{
\Gamma \vdash_{*} F : (\Pi[x : A]. B)
&
\Gamma \vdash_{*} V : A
&
\text{x is free for V in B}
}
\]

\[
\infer[\m{abs}]
{
\Gamma \vdash_{*} (\lambda (x : A) . F) : (\Pi (x : A) . B)
}
{
\Gamma , x : A\vdash_{*} F : B
&
\Gamma \vdash_{*} (\Pi (x : A) . B) : K
}
\]

\[
\infer[\m{abs*}]
{
\Gamma \vdash_{*} (\lambda [x : A] . F) : (\Pi [x : A] . B)
}
{
\Gamma , x : A\vdash_{*} F : B
&
\Gamma \vdash_{*} (\Pi [x : A] . B) : K
&
x \notin FV(F*)
}
\]

\[
\infer[\m{conv}]
{
\Gamma \vdash_{*} A : B'
}
{
\Gamma \vdash_{*} A : B
&
\Gamma \vdash_{*} B* \equiv_{\beta\eta\alpha*} B'*
&
\Gamma \vdash_{*} B' : K
}
\]

\caption{Typing rules for $ICC*$}
\label{icc*:rules}
\end{figure}

Barras showed consistency for the rules in 
this Church-style $ICC*$  calculus
by consistency of $ICC$ \citep{barras2008implicit}.  

It is important to note that the conversion rule given by Barras
is more general than is necessary, as it only requires equality of 
the projected forms.

The abstraction rule is also more specific than is needed, as it 
prevents proof relevant implicit terms.  
While removing this restriction obfuscates the connection to $ICC$, 
it is possible to instead project from $ICC*$ directly into the 
Church-style $CC$ to obtain consistency, using the same proof
techniques.

I now define the system $ICC^+$, which shares the syntax of $ICC*$.

\begin{figure}[H]
\[ 
V^+ := V
\]

\[
(E_1\;E_2)^+ := E_1^+ \; E_2^+
\]

\[
(E_1\;[E_2])^+ := E_1^+\; E_2^+
\]

\[
(\lambda (V : T). E )^+ := \lambda V : T^+. E^+
\]

\[
(\lambda [V : T]. E )^+ := \lambda V : T^+ . E^+
\]

\[
(\Pi [V : T]. E )^+ := \Pi V : T^+ . E^+
\]

\[
(\Pi (V : T). E )^+ := \Pi V : T^+. E^+
\]
\caption{Projection from $ICC^+$ to $CC$}
\label{icc+:proj}
\end{figure}

The typing rules for $ICC^+$ are the same as those for $ICC*$ 
but for the modification mentioned earlier. 
Note that these rules no longer depend on the projection.  


\begin{figure}[H]
\[
\infer[\m{abs}*]
{
\Gamma \vdash_+ (\lambda [x : A] . F) : (\Pi [x : A] . B)
}
{
\Gamma , x : A\vdash_+ F : B
&
\Gamma \vdash_+ (\Pi [x : A] . B) : K
}
\]

\[
\infer[\m{conv}]
{
\Gamma \vdash_+ A : B'
}
{
\Gamma \vdash_+ A : B
&
\Gamma \vdash_+ B \equiv_{\beta\eta\alpha*} B'
&
\Gamma \vdash_+ B' : K
}
\]
\caption{Altered rules for $ICC^+$}
\label{icc+:rules}
\end{figure}

The projection defined in \ref{icc+:proj} is used to convert
typing derivations for $ICC^+$ to $CC$ in order to show consistency 
of $ICC^+$.


\begin{theorem}
Soundness of extraction:

\[
\Gamma \vdash_+ M : T \implies \Gamma+ \vdash M+ : T+
\]

\label{icc+:sound}
\end{theorem}

Note that this \ref{icc+:sound} can be proved trivially
by structural induction, and replacing rules ending in * 
with their counterparts.

\begin{definition}
$ \m{Term}_{ICC^+}  = \{ M : \exists T,\Gamma . \Gamma \vdash_{+} M : T \}$
\end{definition}

\begin{theorem}
\textbf{(Strong Normalization)} $\forall M \in \m{Term}_{ICC^+}. SN(M)$
\label{icc+:cons}
\end{theorem}

***************
Iffy! Can also do a reduction to ECC, where we have products, but we loose $\eta$ 
expansion normalization. \citep{luo1989ecc}
***************
This is a direct result of the consistency of $CC$ and 
the soundness result, \ref{icc+:sound}.

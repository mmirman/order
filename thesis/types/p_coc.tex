\section[$CC$]{The Calculus of Constructions}
% -----------------------------------------

\subsection[Overview]{Overview}

\begin{frame}
\frametitle{The Calculus of Constructions}
\begin{itemize}
\item Defined by Coquand \citep{coquand1986calculus}.
\item A pure type system
\item When extended with universes, a good theorem proving language.

\end{itemize}
\end{frame}

% -----------------------------------------

\begin{frame}
\frametitle{Pure Type Definition}

\begin{definition}
\textbf{(PTS for $CC$)}

\begin{align}
A &= \{ *, \Box \}
\\
S &= \{ (* : \Box) \}
\\
R &= \{ (*,*,*),(*,\Box,\Box),(\Box,\Box,\Box),(\Box,*,*)\}
\end{align}  
\end{definition}

\end{frame}

% -----------------------------------------
\subsection{Properties}

\begin{frame}
\frametitle{Properties}

\textbf{(PTS for $CC$)}

\begin{itemize}
\item Terms can depend on types
\item Terms can depend on terms
\item Types can depend on terms
\item Types can depend on types
\end{itemize}

\end{frame}


% -----------------------------------------
\begin{frame}

\frametitle{Notation}

\begin{definition}
If $\Gamma \vdash_{CC} P : T : K$ means $\Gamma \vdash_{CC} P : T$ and $\Gamma \vdash_{CC} T : K$
\end{definition}


\begin{definition}
$ \m{Term}_{CC}  = \{ M : \exists T,\Gamma . \Gamma \vdash_{cc} M : T \}$
\end{definition}

\end{frame}

% -----------------------------------------

\begin{frame}
\frametitle{Consistency}

Strong normalization in the Calculus of Constructions implies consistency.

\begin{theorem}
\textbf{(Strong Normalization)} $\forall M \in \m{Term}_{CC}. SN(M)$
\label{cc:cons}
\end{theorem}

The easiest proof is due to Geuvers \citep{Geuvers94ashort} 

\end{frame}

% -----------------------------------------

\begin{frame}
\frametitle{Impredicativity}

\begin{itemize}
\item Small types can be generalized over small types
\item Useful for general logic programming libraries.
\item Restrictive for metaprogramming (type is not a type).
\item Extended Calculus of Constructions due to \citet{luo1989ecc} 
\item Cumulative universe inference employed in Caledon \citep{callaghan2001implementation}.
\end{itemize}

\end{frame}


% -----------------------------------------
\subsection{Theorems in Caledon}
\begin{frame}
\frametitle{Theorems in Caledon}

\begin{itemize}
\item Caledon, programs might not terminate. 
\item Caledon programs output consistent theorems in the Calculus of Constructions.
\end{itemize}

\begin{definition}
In the Caledon language, $\m{prop} = *$ and $\m{type} = \Box$
\end{definition}

\begin{definition}
If $\Gamma \vdash_{CICC} P : T : \m{prop}$ in the caledon language, then $T$ is a theorem, and $P$ is a proof.
\end{definition}

\end{frame}

The Implicit Calculus of Constructions (ICC) \citep{pollack1990implicit}, 
is an extention to the standard Calculus of Constructions which allows
for declaration that in all uses of a function, the argument be omitted 
and chosen during typechecking based on a provability relation.

Standard CC, and even standard LF 
can be an unnecessarily verbose languages as seen in the example \ref{code:long}.

\begin{figure}[H]
\begin{lstlisting}
defn churchList : prop -> prop
  as \ A : prop . [lst : prop -> prop] ([C] lst C) -> (A -> [C] lst C -> lst C) -> [C] lst C

defn mapCL : [A : prop] [B : prop] (A -> B) -> churchList A -> churchList B
  as \ A    : prop                    . 
     \ B    : prop                    .
     \ F    : A -> B                  . 
     \ cl   : churchList A            .
     \ lst  : prop -> prop            .
     \ nil  : [B] lst B               .
     \ cons : B -> [B] lst B -> lst B .

          cl lst nil (\v . cons (F v))

defn mapResult : churchList natural
  as mapCL natural boolean (\ a : natural . isZero a) someList

\end{lstlisting}
\caption{Maping over the church encoding of a list}
\label{code:long}
\end{figure}

Ideally, one would like to omit the types that valuesa are parameterized over when they are redundant
and can be infered from context. 

Omiting these types gives rise to the notion of an implicit type system.  
The Hindley-Milner \citep{hindley1969principal} system for inferring principle types in system F
is a special case of the system where implicit universally quantified type variables are automatically
resolved.

The type system of caledon is based on three different formalisms for working with implicit arguments:  
the Bicolored Calculus of Constructions ($CC^{Bi}$) \citep{luther2001more}, 
the Implicit Calculus of Constructions (ICC) \citep{miquel2001implicit}, 
and the Caledon Implicit Calculus of Constructions (CICC).

The last is intended to be a combination of the first two, or a partial 
erasure system for the Bicolored Calculus of Constructions, 
which allows for optional specialization.
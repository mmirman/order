\section{Caledon Implicit Calculus of Constructions}


In these sections I lay out the type system for Caledon. I first describe the target system,
on which has meaningful theorems, $CICC$. 
This is the interpreter output of successfully running a Caledon program. 
I then describe the $CICC$ syntax and statics. 
Caledon’s exposed type system is a variant of the Implicit Calculus of Constructions, ICC [53],
which for the rest of the paper I will refer to as $CICC^-$. The $CICC^-$ statics are realized
by the interpretation of the unification problem form ($UPF$) generated by elaboration.
The syntactic pipeline is as follows:


\[
CICC^- \rightarrow UPF \rightarrow CICC
\]

Proving the consistency of $CICC$ requires a further elaboration into $CC$

\begin{figure}[H]
\[ 
E ::= 
V 
\orr S 
\orr E\;E 
\orr \lambda V : T. E 
\orr ?\lambda V : T. E 
\orr \Pi V : E . E 
\orr ?\Pi V : E . E 
\orr E \{ V : E = E \}
\]

\caption{Syntax of $CICC$}
\label{cicc:syntax}
\end{figure}

The \textit{dependent} explicit and implicit products are written $\Pi v : E . E $ and $?\Pi v : A . E$. 
The \textit{non-dependent} explicit and implicit products are written $T \rightarrow T$ 
and $T \Rightarrow T$ respectively.

Note, that in $CICC$ system, $?\lambda x . A \not\equiv_\alpha ?\lambda y . [x / y] A$ 
and $?\Pi x . A \not\equiv_\alpha ?\Pi y . [x / y] A$  if $x \neq y$.  
This implies that the behavior of Caledon's implicit argument is something akin to a structural dependent product, in that it binds a name rather than a variable.   

Before the typing rules of this system can be given, the notion of a constrained name of a term must be defined.

\begin{definition}
The constrained names on a term, written $CN(M)$ is a set defined as follows:

\begin{align}
CN(M \{ x = E \}) &= \{ x \} \cup CN(M)
\\
CN(\m{otherwise}) &= \emptyset
\end{align}

\end{definition}

The constrained names on a term are defined to be the constraints placed at the top level on 
a term.  For example, in the term $\m{some_pred} \; \m{type} \; \m{nat} \{ A = nat \} \{ B = nat \}$ the constrained names are $A$ and $B$.  

\begin{definition}
The generalized names for a term, written $GN(M)$ is a set defined as follows:

\begin{align}
GN(?\Pi x : T . M) &= \{ x \} \cup GN(M) \cup GN(T)
\\
GN(\m{otherwise}) &= \emptyset
\end{align}
\end{definition}

The definition of generalized names and of constrained names are in a way complimentary, in that
you may only the generalized names of the type of a term may be constrained.

\begin{definition}
The bound names for a term, written $BN(M)$ is a set defined as follows:

\begin{align} 
BN(?\lambda x : T . M) &= \{ x \} \cup BN(M) \cup BN(T)
\\
BN(\m{otherwise}) &= \emptyset
\end{align}

\end{definition}

As defined above, bound names and bound variables can no longer be treated the same in the semantics.  
Specifically, $?\lambda x : A . B$ does not have the same semantics as $?\lambda y : A . [y / x] B$.  
This implies that alpha conversion is now severely limited.  

There are a few ways to deal with this.  
The most attractive possibility is to interpret names as a kind of record modifier.
This can be seen as saying $\{ x : T = N \} : \{ x : N \}$, 
and $?\lambda x : N . B$ is really just $\lambda y : \{ x : N \} . [ y.x / x ] N$ where $ .x : \{ x : N \} \rightarrow N$.

\begin{figure}[H]
\begin{lstlisting}
defn nat : prop 
  as [a : prop] a -> (a -> a) -> a

defn nat_1 : nat -> prop
  as \ N : nat . [a : nat -> prop] a zero -> succty a -> a N

defn rec : nat -> prop -> prop
  as \ nm : nat . \ N : kind . nat_1 nm * N

defn get : [N : kind] [nm : nat] nat_1 nm * N -> N
  as \ N : kind . \nm : nat . \ c : (string2 nm, N) . snd c

defn put : [N : kind] [nm : nat] nat_1 nm ->  N -> nat_1 nm * N
  as \ N : kind . \nm : nat . \nmnm : nat_1 nm . \ c : N . pair nmnm N
\end{lstlisting}
\caption{Definitions for extraction, written in $CC$ with Caledon syntax}
\label{code:ideal}
\end{figure}

We can further convert this into traditional dependent types by constructing
 type invariants as seen in \ref{code:ideal}.

Then $?\lambda x : A . B$ and $?\Pi x : A . B$ 
becomes $\lambda y : \m{rec}\;\bar{x}\;A . [get A \bar{x} y / x ] B$
and $?\Pi y : \m{rec}\;\bar{x}\;A . [get A \bar{x} y / x ] B$.

Similarly, $N \{ x\; : \; T \; = \;A \} $
would become $N\;( \m{put} \; T \; \bar{x} \; \bar{\bar{x}} \; A)$.

One might notice that $N$ in $\m{get}$ is of type $\m{kind}$.  
In simple $CC$, this is unfortunately not an actual type. 
Rather, it refers to the use of either $\m{type}$ or $\m{prop}$.  
Allowing $N : \m{type}$ is not permited in the standard $CC$ since it is the subject of quantification.
This is possible in $CC_\omega$ however.
In this case, $\m{kind}$ would always refer to the next universe after the highest
universe mentioned in the program.
In the Caledon language implementing simple $CC$ we are free to define 
$\m{kind}$ as $\m{type}_1$, since it will always be larger than any
type or kind mentioned in a Caledon program.  
Fortunately, Geuvers' proof \citep{geuvers1993logics} of strong normalization in the presence of 
$\eta$ conversion applies to the Calculus of Construction with one impredicative universe and two predicative
universes.

This intuitive conversion leads to the following typing rules for $CICC$.

\begin{definition}
\textbf{(Typing for $CICC$)}

\[ \begin{array}{lr}
\infer[\m{wf/e}]
{
\cdot \vdash_{ci} 
}{}
&
\infer[\m{wf/s}]
{
\Gamma, x : T \vdash_{ci} 
}
{
\Gamma \vdash_{ci} T : s
&
x \notin DV(\Gamma)
}
\end{array} \]

%% axioms %%
%%%%%%%%%%%%
\[
\infer[\m{axioms}]
{
\Gamma \vdash_{ci} c : s
}
{
\Gamma \vdash_{ci}
&
(c,s) \in A
}
\]

%% start %%
%%%%%%%%%%%
\[
\infer[\m{start}]
{
\Gamma,x:A \vdash_{ci} x :A
}
{
\Gamma \vdash_{ci} A:s
&
s \in S
}
\]

%% prod %%
%%%%%%%%%%
\[
\infer[\m{prod}]
{
\Gamma \vdash (\Pi x : A . B) : s_3
}
{
\Gamma \vdash A : s_1
&
\Gamma,x:A \vdash B : s_2
&
(s_1,s_2,s_3) \in R
}
\]

%% prod* %%
%%%%%%%%%%%
\[
\infer[\m{prod}*]
{
\Gamma \vdash (?\Pi x : A . B) : s_3
}
{
\Gamma \vdash A : s_1
&
\Gamma,x:A \vdash B : s_2
&
(s_1,s_2,s_3) \in R
}
\]

%% gen %%
%%%%%%%%%
\[
\infer[\m{gen}]
{
\Gamma \vdash_{ci} \lambda x : T . M : (\Pi x : T . U)
}
{
\Gamma , x : T \vdash_{ci} M : U
&
\Gamma \vdash_{ci} (\Pi x : T . U) : s
&
s \in S
&
x \notin FV(M) \cup BN(M) \cup GN(U)
}
\]

%% gen* %%
%%%%%%%%%%
\[
\infer[\m{gen}*]
{
\Gamma \vdash_{ci} ?\lambda x : T . M : (?\Pi x : T . U)
}
{
\Gamma , x : T \vdash_{ci} M : U
&
\Gamma \vdash_{ci} (?\Pi x : T . U) : s
&
s \in S
&
x \notin FV(M) \cup BN(M) \cup GN(U)
}
\]

%% app %%
%%%%%%%%%
\[
\infer[\m{app}]
{
\Gamma \vdash_{ci} M N : U [N/x]
}
{
\Gamma \vdash_{ci} M : \Pi x : T . U
&
\Gamma \vdash_{ci} N : T
}
\]

%% inst/b %%
%%%%%%%%%%%%
\[
\infer[\m{inst/b}]
{
\Gamma \vdash_{ci} M \{ x : T = N \} : U [N/x]
}
{
\Gamma \vdash_{ci} M : ?\Pi x : T . U
&
\Gamma \vdash_{ci} N : T
& 
x \notin GN(M)
&
x \notin BN(U)
}
\]

\label{cicc:typing}
\end{definition}


\subsection{Main Results}

Most of the theorems relating to $CICC$ can be obtained by a simple projection into $CC$.

\newcommand{\CICCproj}[1]{ \left\llbracket #1 \right\rrbracket_{ci}}

\begin{definition}
\textbf{ (Projection from $CICC$ to $CC$) }

\[
\CICCproj{v} := v
\]

\[
\CICCproj{s} := s
\]

\[
\CICCproj{E_1 \; E_2} := \CICCproj{E_1} \; \CICCproj{E_2}
\]

\[
\CICCproj{E_1 \; \{ x : T = E \}} := \CICCproj{E_1} \; (\m{put}\;\CICCproj{T}\;\bar{x}\; \bar{\bar{x}}; \CICCproj{E_2} )
\]

\[
\CICCproj{\lambda v : T . E } := \lambda v : \CICCproj{T} . \CICCproj{E}
\]

\[
\CICCproj{?\lambda v : T . E } := \lambda y : \m{rec}\;\bar{v}\; \CICCproj{T} . \CICCproj{ [ \m{get}\; \CICCproj{T}\; \bar{v}\; y  / v ] E}
\;\text{ where $y$ is fresh}
\] 

\[
\CICCproj{\Pi v : T . E } := \Pi v : \CICCproj{T} . \CICCproj{E}
\]

\[
\CICCproj{?\Pi v : T . E } := \Pi y : \m{rec}\;\bar{v}\;\CICCproj{T} . \CICCproj{ [ \m{get}\;\CICCproj{T}\; \bar{v}\; y  / v ] E}
\;\text{ where $y$ is fresh}
\]

\label{cicc:proj}
\end{definition}

It is significant that church numerals be used for the representation of the name in the record, 
as no extra axioms need to be included in the context of the translation for the translation to be valid.  
This necessity is seen in \ref{ci:sound}.

\begin{lemma}

If $\Gamma \vdash_{ci} A : T$ then $\Gamma \vdash_{ci}$

\label{ci:wfctxt}
\end{lemma}

\begin{lemma}

If $\Gamma \vdash_{ci} A : T$ then $\Gamma \vdash_{ci} T : s$ for some sort $s$

\label{ci:wtt}
\end{lemma}

\begin{theorem}

\textbf{(Soundness of extraction)}  

\begin{alignat}{4}
\Gamma &\vdash_{ci}&  & \implies & \CICCproj{\Gamma} & \vdash_{cc} &
\\
\Gamma &\vdash_{ci}& A : T & \implies & \CICCproj{\Gamma} & \vdash_{cc} & \CICCproj{A} : \CICCproj{ T }
\end{alignat}

\label{ci:sound}
\end{theorem}

It is easy to see how this proof follows from the 
definition \ref{cicc:proj}, and the two lemmas \ref{ci:wfctxt} and \ref{ci:wtt}.

\begin{definition}
$ \m{Term}_{ci}  = \{ M : \exists T,\Gamma . \Gamma \vdash_{ci} M : T \}$
\end{definition}

\begin{theorem}
\textbf{(Consistency)}  $\not \exists M \in \m{Term}_{ci}. \vdash M : \Pi x : \m{prop} . x$
\label{ci:cons}
\end{theorem}

This follows clearly from the soundness of the extraction, 
\ref{ci:sound}, the consistency of $CC$ \ref{cc:cons}, and the fact that we 
have obviously more possible reductions in $CC$ than in $CICC$.  

Before we can prove the strong normalization theorem, we need to show that reductions in the extracted calculus 
correspond to reductions in $CICC$. 

\begin{theorem}
\textbf{(Semantic Equivalence)} 
$\forall M \in \m{Term}_{cicc}$ such that $\CICCproj{M} \rightarrow_{\beta\eta*} N'$
implies 
$\exists M' \in \m{Term}_{cicc}$ such that $M \rightarrow_{\beta\eta\alpha} M'$ and $\CICCproj{M'} \equiv N'$
\label{ci:se}
\end{theorem}

While the proof here is somewhat technical, the result is obvious, as the extraction and calculus
have been written so as to ensure that the result is true.

\begin{theorem}
\textbf{(Subject Reduction)} If $\Gamma \vdash_{ci} M : T$ and $M \rightarrow_{\beta\eta\alpha*} M'$ then $\Gamma \vdash_{ci} M' : T$
\end{theorem}

In non dependent type theories, this theorem is rather trivial.  However, in $CC$, this theorem depends heavily on the Church-Rosser 
theorem.  Fortunately, $CICC$ can be considered a bicoloring of the syntax of $CC$, and the proof of subject reduction 
for $CC$ can be leveraged to $CICC$ using a forgetfull extraction.

\begin{theorem}
\textbf{(Strong Normalization)} $\forall M \in \m{Term}_{ci}. SN(M)$
\label{ci:sn}
\end{theorem}

That we can cleanly translate into the calculus of constructions without loss or gain of 
semantical translation implies strong normalization for $CICC$ with $\beta\eta$ equivalence, 
and $\eta$ expansion.  This is the most important theorem of the section, as it implies that typchecking a Caledon 
statement will allow that statement to be compiled to pattern form to be used in proof search.  
Without this theorem typechecking is a weak property and valid programs might not be useable.  
While type safety in the presence of the strong normalization theorem does not ensure bounded proof search, 
it does ensure bounded unification.
The strong normalization theorem is to logic programming languages as 
the progress theorem is to functional programming languages.

That $CICC$ is simply an extention of $CC$ and not a modification of $CC$ implies we have the completeness theorem, \ref{ci:comp}.

\begin{theorem}
\textbf{(Completeness)}  $\forall M,T \in \m{Term}_{cc}. \vdash_{cc} M : T \implies \vdash_{ci} M : T$
\label{ci:comp}
\end{theorem}

This theorem is trivial since the syntax of $CC$ is a subset of the syntax of $CICC$.

\section{Caledon Implicit Calculus of Constructions}

Caledon's type system is a varient of ICC and $CC^{Bi}$, 
which for the rest of the paper I will refer to as CICC.  
This sytem contains two products and two binders - one each for implicit and explicit arguments. 
Unlike ICC, there is no rule that allows for an unmarked value to obtain an implicit product type, which
makes type checking somewhat simpler.  
In addition, there is a new form of application to allow for the explicit
selection of an implicit argument to constrain.

\begin{figure}[h]
\[ 
E ::= 
V 
| S 
| E\;E 
| \lambda V . E 
| ?\lambda V . E 
| \Pi V : E . E 
| ?\Pi V : E . E 
| E \{ V := E \}
\]

\caption{Syntax of CICC}
\label{cicc:syntax}
\end{figure}

The \textit{non-dependent} explicit and implicit products are written $T \rightarrow T$ 
and $T \Rightarrow T$ respectively.


Note, that in CICC system, $?\lambda x . A \neg\equiv_\alpha ?\lambda y . [x / y] A$ 
and $?\Pi x . A \neg\equiv_\alpha ?\Pi y . [x / y] A$  if $x \neq y$.  This implies that the 
behavior of caledon's implicit argument is something akin to a structural dependent product.  


Before the typing rules of this system can be given, the notion of a constrained name of a term must be defined.

\begin{definition}
The constrained names on a term, written $CN(M)$ is a set defined as follows:

\[ 
CN(M \{ x := E \}) = \{ x \} \cup CN(M)
\]

\[ 
CN(\m{otherwise}) = \emptyset
\]

\end{definition}

\begin{definition}
The generalized names for a term, written $GN(M)$ is a set defined as follows:

\[ 
GN(?\Pi x . M) = \{ x \} \cup CN(M)
\]

\[ 
GN(\m{otherwise}) = \emptyset
\]

\end{definition}

\begin{definition}
The bound names for a term, written $GN(M)$ is a set defined as follows:

\[ 
GN(?\lambda x . M) = \{ x \} \cup CN(M)
\]

\[ 
GN(\m{otherwise}) = \emptyset
\]

\end{definition}


\begin{figure}[h]

\[
\infer[\m{gen}]
{
\Gamma \vdash ?\lambda x . M : (?\Pi x : T . U)
}
{
\Gamma , x : T \vdash M : U
&
\Gamma \vdash (?\Pi x : T . U) : s
&
s \in S
&
x \notin FV(M) \cup BN(M) \cup GN(U)
}
\]


\[
\infer[\m{inst-free}]
{
\Gamma \vdash M : U [N/x]
}
{
\Gamma \vdash M : ?\Pi x :T . U
&
\Gamma \vdash N : T
& x \notin CN(M)
}
\]


\[
\infer[\m{inst-bound}]
{
\Gamma \vdash M \{ x := N \} : U [N/x]
}
{
\Gamma \vdash M : ?\Pi x :T . U
&
\Gamma \vdash N : T
& 
x \notin GN(M)
&
x \notin BN(U)
}
\]


\[
\infer[\m{imp-prod}]
{
\Gamma \vdash (\forall x : A . B) : s_3
}
{
\Gamma \vdash A : s_1
&
\Gamma,x:A \vdash B : s_2
&
(s_1,s_2,s_3) \in R
}
\]


\[
\infer[\m{strength}]
{
\Gamma \vdash M : U
}
{
\Gamma , x : T \vdash M : U
&
x \notin FV(M) \cup FV(U)
}
\]

\[
\infer[\m{ext}]
{
\Gamma \vdash M : T
}
{
\Gamma\vdash \lambda x . (M x)  : T 
&
x \notin FV(M)
}
\]
\caption{Typing for ICC}
\label{cicc:typing}
\end{figure}

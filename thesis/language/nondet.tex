\section{Controlled Nondeterminism}

A logic program need not only be a deterministic depth first pattern search. For purely declarative 
axioms, the depth first strategy is usually, in fact, incomplete 
and not representative of what should and should not halt.
Depth first search, however, can be more efficient in many cases and when used intentionally
will constrict nondeterminism. Concurrency in a program with IO has
well known uses and advantages. The breadth first proof search strategy can conveniently
be represented by a program where every pattern-match forks and then executes
concurrently. This is not ideal if used indiscriminately. An ideal implementation
allows one to control the patterns that are searched in parallel and in sequence. 
In the following snippet of the code, the distinction between
breadth first and depth first queries are used to emulate the concept of concurrency and
cause the program to have more complex behavior.

\begin{figure}[H]
\begin{lstlisting}

query main = runBoth false

defn runBoth : bool -> type
  >| run0 = runBoth A
            <- putStr ``ttt ``
            <- A =:= true
   | run1 = runBoth A 
            <- putStr ``vvvv''
            <- A =:= true
   | run2 = runBoth A
            <- putStr ``qqqq''
            <- A =:= true
  >| run3 = runBoth A
            <- putStr `` jjj''
            <- A =:= false

\end{lstlisting}
\caption{Nondeterminism control}
\label{nondet:ex1}
\end{figure}

In example \ref{nondet:ex1}, the query main prints to the screen something similar to “ttt
vqvqvqvq jjj”. This happens because, despite proof search failing on the first three axioms due
to an incorrect match, the fail is deferred until after IO has been performed. The middle
axioms, ``run1'' and ``run2'' are declared to be breadth first axioms, while ``run0'' and ``run3'' are
declared to be depth first axioms.  
The declaration of an axiom as being depth first implies that it's entire tree must be searched for a successful
proof before the next axiom can be attempted.  While breadth first axioms make no such guarantee, 
it is also not guaranteed that they will run concurrently.  
If for example they both make calls to predicates which each only have a single depth first axiom, they will not be
any more concurrent than before.  


\begin{figure}[H]
\begin{lstlisting}

defn fail : type
  as true =:= false

defn then : type -> type -> type
  >| then-A = Fst `then` Snd <- Fst <- fail
  >| then-B = Fst `then` Snd <- Snd

query main = print ``Hello `` `then` print ``world!''

\end{lstlisting}
\caption{Sequential predicate}
\label{nondet:seq}
\end{figure}

The predicate “then” in figure \ref{nondet:seq} executes its first routine and subsequently its second sequentially.
To understand how this works, it is helpful to step through the query
“main”. When “main” is called, it will first initiate a goal of the form “then (print ‘Hello
’) (print ‘world!’)”. Because the axioms “then-A” and “then-B” are both preceded by ``>|'',
the “then-A” is attempted, the search of which is constrained to be entire. Since “(print
’Hello ’)” succeeds and does not nondeterministically branch, “fail” will be initiated as
the next goal. ``fail'' will certainly fail, and the current branch will end. “then-B” is the
next attempt. It will succeed since “(print ’world!’)” succeeds.

\begin{figure}[H]
\begin{lstlisting}

defn while : type -> type -> type
   | while-A = Fst `while` Snd <- Fst
   | while-B = Fst `while` Snd <- Snd

query main = print ``aaaa'' `while` print ``bbbb''

\end{lstlisting}
\caption{Concurrent predicate}
\label{nondet:con}
\end{figure}


The predicate “while” in figure \ref{nondet:con} executes its first routine at the same time as its
second. The result of “main” might then be “abababab”. It is important to note that in this notion of concurrency, neither thread is the ``original'' thread.

The Caledon language implementation has a few unique properties, not all related to
the exposed logic of the language. In this chapter I discuss the some of the details of
the specification of the Caledon language, and its implementation. The algorithm for
actually performing higher order unification and type inference is unusual in that it
uses a zipper style context implemented by a finger tree based sequence, and does not
perform linear passes on the unification problem. Families are introduced as a notion
of a coherent set of axioms for proof search, falling under the same searchable family.
Nondeterminism control is also discussed as a way of letting the programmer choose
between sequential and concurrent execution, or efficient or complete searches. Methods
of interacting with the world also need to be defined.


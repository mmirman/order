\section{Type Inference}

While the generation of unification problems in Caledon is straightforward as described in the previous
sections, the implementation of the higher order pattern unification algorithm
is convoluted. If unification were to be implemented in Ollibot \citep{pfenning2009substructural}, a much simpler, 
yet significantly less efficient implementation might be possible.

Universally quantified and lambda quantified variables 
are easily represented by DeBruijn indexes, since no universal quantifiers are ever
introduced between two preexisting universal quantifiers. On the other hand, existential
variables are introduced at any location. Rather than complicating substitutions,
existential variable instances are represented by unique names with depth indexes, instead of height indexes.
Because existential variables are not explicitly named in the context, each existential variable instance carries its
own type, lifted to the location of the instance.

Since such a representation requires traversals and modifications of a tree structure,
a natural solution is to use a zipper to represent the entire structure. Because traversing the structure
downward also builds a context of universal variables and passed conjunctive
paths, the same zipper structure can likewise be used as the type context for DeBruijn variable
lookup. In an imperative language with effects, one might think that using
a vector to hold the zipper context of the unification problem is optimal. However,
since nondeterminism is essential to proof search, complications
arise if the structure is shared among threads manually.  A pure data structure based on
finger trees \citep{hinze2006finger} known as a sequence turns out to be an 
ideal choice of structure. Concatenation in this structure is constant time, and splitting and lookup are logarithmic.
While logarithmic lookup time is slightly worse than the constant lookup time for a vector, its benefit is
that it automatically shares relevant unchanged sections between threads.


    

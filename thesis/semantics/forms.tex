\section{Forms for Unification}

\begin{definition}
Spine Form
\begin{align}
N &::= P
   \orr \lambda V : N . N 
\\
P &::= V 
  \orr P N 
\end{align}
\label{def:spine}
\end{definition}

We write $\Pi V : N . P$ as a synonym for 
$\Pi\; N \; (\lambda V : N . P)$ in the rest of this thesis.
This simplifies the presentation of the unification algorithm, 
as then $\Pi$ can be considered a traditional constructor
that can also be used to direct the unification procedure.

Spine terms have the incredibly useful property of always being in head normal form, 
meaning that the head of every term is a constructor, 
and every argument is either a constructor or lambda term.

\subsection{Higher Order Patterns}

While spine form is restrictive enough that its terms are always in head normal form, 
it is not restrictive enough for unification problems to be decidable.  
Miller \citep{miller1991logic} showed that for any unification instance given in 
the pattern fragment shown in \ref{def:pattern}, unification is decidable.  

Pattern form is specified with respect to partial permutations $\phi$, 
which are injective mappings from finite domains to finite domains.

\newcommand{\Pat}{\;\m{ Pat }\;}
\begin{definition}
Pattern Form:  Note that $\Delta$ is the existential context and 
$\Gamma$ is the universal context.

\[
\infer[\m{P/ABS}]{
\Delta ; \Gamma \vdash \lambda u : A . M \Pat
}{
\Delta ; \Gamma \vdash A \Pat
&
\Delta ; \Gamma,u \vdash M \Pat
} \]


\[ \begin{array}[2]{lr}
\infer[\m{P/CON}]{
\Delta ; \Gamma \vdash c \;M_1\cdots M_m \Pat
}{
\Delta ; \Gamma \vdash M_1 \Pat
&
\cdots
&
\Delta ; \Gamma \vdash M_m \Pat
}
&
\infer[\m{P/VAR}]{
\Delta ; \Gamma \vdash u \;M_1\cdots M_m \Pat
}{
\Delta ; \Gamma \vdash M_1 \Pat
&
\cdots
&
\Delta ; \Gamma \vdash M_m \Pat
&
u \in \Gamma
}
\end{array} \]


\[ \begin{array}[2]{lr}
\infer[\m{P/PROP}]{
\Delta ; u_1 ,\cdots u_p 
\vdash x \;u_{\phi(1)}\cdots u_{\phi(m)} \Pat
}{
\phi \text{ is a partial permutation}
&
x \in \Delta
}
&
\infer[\m{P/VAR}]{
\Delta ; \Gamma \vdash M \Pat
}{
\Delta ; \Gamma \vdash M' \Pat
&
M \equiv_{\eta} M'
}
\end{array} \]

\label{def:pattern}
\end{definition}


\subsection{Canonical Forms}

The unification algorithm in \citet{pfenning1991unification} for the 
``Calculus of Constructions'', which the meta-theory of Caledon is based on, 
relies on expressions being presented in 
\textit{$\beta$-normal $\eta$-long form} (or \textit{canonical form}), 
meaning that they are $\eta$ expanded to conform to their type signature.  
In the initial publication of this paper, it was taken as a hypothesis that 
every well-typed term in $CC$ has a unique $\beta$-normal $\eta$-long form.  This is now known
to be the case \citep{abel2010towards}.

\newcommand{\FormFor}{\;\Rightarrow\;}
\begin{definition}
Canonical Forms

\[ \begin{array}{lr}
\infer[\m{F/ax}]{
\Gamma \vdash s_1 \FormFor s_2
}{
(s_1,s_2) \in A
}
&
\infer[\m{F/prod}]{
\Gamma \vdash \Pi x : A . B \FormFor s_3
}{
\Gamma \vdash A \FormFor s_1
&
\Gamma, x : A \vdash B \FormFor s_2
&
(s_1,s_2,s_3) \in R
}
\end{array} \]

\[
\infer[\m{F/lam}]{
\Gamma \vdash \lambda x : A . M \FormFor \Pi x : A . B
}{
\Gamma,x : A\vdash M \FormFor B
&
\Gamma\vdash A \FormFor s
} 
\]

\[
\infer[\m{F/app}]{
\Gamma \vdash h\;M_1 \cdots M_n \FormFor D
}{
\Gamma \vdash h\;M_1 \cdots M_n : D
&
\Gamma \vdash M_1 \FormFor A_1
&
\cdots
&
\Gamma \vdash M_n \FormFor A_n
&
} 
\]
where $D$ is atomic
\label{def:canonical}
\end{definition}

It has been proven that the standard ``Calculus of Constructions'' is 
$\beta$-normal $\eta$-long form strongly normalizing.  
Unfortunately, normalization into this form is not possible without type 
information.  Later, a typed substitution algorithm will be given which 
ensures normalization into this form.  

The notions of canonical form and of a higher order pattern are also trivially
extensible into Church-style $CC^{Bi}$ (ie, ICC* from \citep{barras2008implicit}), 
where strong normalization into $\beta$-normal $\eta$-long form is also provable, as is shown by \citet{barras2008implicit}.

\section{Semantics for $CICCI$}


\subsection{Substitution With Implicits}

The formulation of hereditary substitution in the presence of 
implicit arguments is not unlike the presentation of
hereditary substitution without implicit arguments with
additional required checks.

\begin{definition}
\textbf{(Implicit Typed Hereditary Substitution)}


\[
[S / x : A]^n_{\Gamma } (?\lambda y : B . N) := ?\lambda y:B . [S / x : A]^n_{\Gamma, y : B} N
\] 

\[
\eta^{-1}_{?\Pi x : A . B}(N) := ?\lambda x : A . N \; \{ x = \eta^{-1}_A(x) \}
\] since $N$ being typeable by $?\Pi x $ means that $x$ can not appear free in $N$

\[
\m{H}_{\Gamma}(P \downarrow ?\Pi y : B_1 . B_2 , \{ v := N \} ) := P\; \{ v := N \} \downarrow [N/y : B_1]^n_{\Gamma}B_2
\]

\[
\m{H}_{\Gamma} ((?\lambda v : A_1 . N) \uparrow ?\Pi v : A_1 . A_2 , \{ v := P \}) 
:= [P/v]^n_{\Gamma \vdash v : A_1} N \uparrow A_2
\]

\[ 
\m{H}(?\lambda v : T . P \uparrow \_ , A) := ?\lambda v : T . \m{H}(P,A)
\]

\label{def:hered}
\end{definition}



\subsection{Unification With Implicits}

Now we can use the convenient fact that $\Gamma \vdash A  \leq B$ implies $\Gamma \vdash B \leq A$ to 
extend the unification rules provided before to apply to $CICC^-$.

\setcounter{tcase}{0}

\begin{tcase}
?Lam-?Lam-same
\end{tcase}


\[
F[?\lambda n,x : A . M \doteq ?\lambda n,y : A . N ]
\UnifiesTo
F[ \forall x : A . M \doteq [x/y]N ]
\]

Note that in this rule, the external name on the left matches the external name on the right.


\begin{tcase}
?Forall-?Forall-same
\end{tcase}

\[
F[?\Pi n,x : A . M \doteq ?\Pi n,y : A' . N ]
\UnifiesTo
F[ A \doteq A' \wedge \forall x : A . M \doteq [x/y]N ]
\]
In this rule the external name on the left must match the external name on the right.


\begin{tcase}
?Lam-?Lam-same
\end{tcase}

This case is distinct from the simple case of equality of dependent products in that an implicit abstraction could lie at the head of 
the spine if it had not already been constrained.

\[
F[(?\lambda n,x : A . M) \; R_1 \cdots R_n \doteq (?\lambda n,y : A' . N) \; R'_1 \cdots R'_m]
\UnifiesTo
F[ A \doteq A' \wedge \forall x : A . H(\cdots H(M, R_1)\cdots R_n)  \doteq H(\cdots H([x/y]N, R'_1)\cdots , R'_n) ]
\]

Again, the external name on the left must match the external name on the right.


\begin{tcase}
?Lam-Unbound
\end{tcase}

if $n \notin BN(N)$ then

\[
F[(?\lambda n,x : A . M) \; R_1 \cdots R_n  \doteq N ]
\UnifiesTo
F[ \exists x : A . H(\cdots H(M, R_1)\cdots R_n) \doteq N \wedge x \in A ]
\]

Here we perform a search for $x \in A$ to satisfy the \verb|inst/f| rule.  Rather than adding a potentially unused constriction to the right hand side, 
we observe that such a constriction could be inferred implicitly.

\begin{tcase}
Uvar-Uvar-BothConst
\end{tcase}

Rather than simply adding a case for universal variables with the possibility of a constriction in the argument list, 
we must modify the already existing case to ``search'' for constrictions on both sides.  
We first have a case which matches constrictions on both sides.

Suppose $n\notin CN(y \; N_1 \cdots N_n) \cup CN(y \; M_1 \cdots M_n)$

\[
F[\forall y : A . G[y \; M_1 \cdots M_{r-1} \;\{ n = A \} \; M_{r} \cdots M_n \doteq y N_1 \cdots N_{r'-1} \; \{ n = A' \} \; N_{r'} \cdots N_n  ]]
\UnifiesTo
F[\forall y : A . G[y M_1 \cdots M_n \doteq y N_1 \cdots N_n \wedge A \doteq A' ]]
\]


\begin{tcase}
Uvar-Uvar-OneConst
\end{tcase}

This case matches when there is only a constriction on one side.

Suppose $n\notin CN(y \; N_1 \cdots N_n) \cup CN(y \; M_1 \cdots M_n)$

\[
F[\forall y : A . G[y \; M_1 \cdots M_{r-1} \;\{ n = A \} \; M_{r} \cdots M_n \doteq y N_1 \cdots N_n  ]]
\UnifiesTo
F[\forall y : A . G[y M_1 \cdots M_n \doteq y N_1 \cdots N_n ]]
\]

In this case we have no reference point to unify $A$ against, and we do not know its type, so we can simply ignore it. 

\begin{tcase}
Uvar-Uvar-Eq
\end{tcase}

If $CN(y \; N_1 \cdots N_n) \cup CN(y \; M_1 \cdots M_n) = \emptyset$

\[
F[\forall y : A . G[y M_1 \cdots M_n \doteq y N_1 \cdots N_n  ]]
\UnifiesTo
F[\forall y : A . G[ M_1 \doteq \wedge N_1 \cdots \wedge M_1 \cdots N_n ]]
\]

This last case behaves exactly as the old ``Uvar-Uvar-Eq'' except that we require there to be no constrained names on either side.

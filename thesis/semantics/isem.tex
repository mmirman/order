\section{Semantics for $CICCI$}


\subsection{Substitution With Implicits}

The formulation of hereditary substitution in the presence of 
implicit arguments is not unlike the presentation of
hereditary substitution without implicit arguments with
additional required checks.

\begin{definition}
\textbf{(Implicit Typed Hereditary Substitution)}


\[
[S / x : A]^n_{\Gamma } (?\lambda y : B . N) := ?\lambda y:B . [S / x : A]^n_{\Gamma, y : B} N
\] 

\[
\eta^{-1}_{?\Pi x : A . B}(N) := ?\lambda x : A . N \; \{ x = \eta^{-1}_A(x) \}
\] since $N$ being typeable by $?\Pi x $ means that $x$ can not appear free in $N$

\[
\m{H}_{\Gamma}(P \downarrow ?\Pi y : B_1 . B_2 , \{ v := N \} ) := P\; \{ v := N \} \downarrow [N/y : B_1]^n_{\Gamma}B_2
\]

\[
\m{H}_{\Gamma} ((?\lambda v : A_1 . N) \uparrow ?\Pi v : A_1 . A_2 , \{ v := P \}) 
:= [P/v]^n_{\Gamma \vdash v : A_1} N \uparrow A_2
\]

\[ 
\m{H}(?\lambda v : T . P \uparrow \_ , A) := ?\lambda v : T . \m{H}(P,A)
\]

\label{def:hered}
\end{definition}



\subsection{Unification With Implicits}

Now we can use the convenient fact that $\Gamma \vdash A  \leq B$ implies $\Gamma \vdash B \leq A$ to 
extend the unification rules provided before to apply to $CICC^-$.

\setcounter{tcase}{0}

\begin{tcase}
?Lam-?Lam-same
\end{tcase}


\[
F[?\lambda n,x : A . M \doteq ?\lambda n,y : A . N ]
\UnifiesTo
F[ \forall x : A . M \doteq [x/y]N ]
\]

Note that in this rule, the external name on the left matches the external name on the right.


\begin{tcase}
?Lam-?Lam-same
\end{tcase}


\[
F[?\lambda n,x : A . M \doteq ?\lambda n,y : A . N ]
\UnifiesTo
F[ \forall x : A . M \doteq [x/y]N ]
\]

Note that in this rule, the external name on the left matches the external name on the right.

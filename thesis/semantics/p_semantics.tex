\section{Semantics}

% -------------------------------------------------------------
\subsection{Overview}
% -------------------------------------------------------------

\begin{frame}
\frametitle{History}
\begin{itemize}
\item Notation presented is based on that given by \citet{pfenning1991logic}.
\item Algorithm extended is based on that given by \citep{pfenning1991unification}.
\end{itemize}
\end{frame}


% -------------------------------------------------------------

\begin{frame}
\frametitle{Higher Order Unification With Search}
\begin{itemize}
\item Find a mapping from existential variables which makes two $CICC^-$ equivalent.
\item Can be decidable for the pattern fragment of terms.
\item We extend with a couple rules for cases like $inst/f$.
\item Unification problems are a target for elaboration.
\end{itemize}
\end{frame}



% -------------------------------------------------------------
\subsection[$UPF$]{Unification Problem Form}

\begin{frame}
\frametitle{Syntax}
\begin{definition}
\textbf{Unification Problem Form}
\[
U ::= U \wedge U 
 \orr \forall V : T . U
 \orr \exists V : T . U 
 \orr U \doteq U
 \orr \top
  \orr T \in T 
  \orr T \in T >> T \in T
\]

\end{definition}

Can be provided meaning with inference rules.

\end{frame}



% -------------------------------------------------------------

\begin{frame}
\frametitle{Implementation}
\begin{itemize}
\item Finger tree zipper datastructure is used to describe the unification problem.
\item Solvable problems are searched in a nearest neighborhood. 
\item Subtrees are blocked until they are changed by substitution.
\end{itemize}
\end{frame}

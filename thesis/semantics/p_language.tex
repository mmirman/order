\section{Language}

% -------------------------------------------------------------
\subsection[Families]{Type Families}
% -------------------------------------------------------------
\begin{frame}
\frametitle{Type Families}
\begin{itemize}
\item Permitting entirely polymorphic axioms at the top level complicates proof search.
\item Axioms grouped by conclusion.
\item Optimizes proof search because it is now possible to limit the search.
\item A bit like ``freeze'' from Twelf.  
\item Mutually recursive definitions are automatically inferred.
\end{itemize}
\end{frame}
% -------------------------------------------------------------

\begin{frame}[fragile]
\frametitle{Type Families Example}

Caledon:

\begin{lstlisting}
defn add : nat -> nat -> nat -> prop
   | addZ = add zero A A
   | addS = add (succ A) B (succ C) 
             <- add A B C
\end{lstlisting}

Twelf:

\begin{lstlisting}
add : nat -> nat -> nat -> type.
addZ : add zero A A.
addS : add (succ A) B (succ C) <- add A B C.
%freeze add.
\end{lstlisting}
\end{frame}


% -------------------------------------------------------------
\subsection[Nondeterminism]{Nondeterminism Control}
% -------------------------------------------------------------
\begin{frame}
\frametitle{Nondeterminism Control}
\begin{itemize}
\item Would like to choose between depth first and breadth first search.
\item Can do this by adding syntax to axioms.
\end{itemize}
\end{frame}
% -------------------------------------------------------------
\begin{frame}[fragile]
\frametitle{Nondeterminism Control Example}

“ttt vqvqvqvq jjj”

\begin{lstlisting}

query main = runBoth false

defn runBoth : bool -> type
  >| run0 = runBoth A
            <- putStr ``ttt ``
            <- A =:= true
   | run1 = runBoth A 
            <- putStr ``vvvv''
            <- A =:= true
   | run2 = runBoth A
            <- putStr ``qqqq''
            <- A =:= true
  >| run3 = runBoth A
            <- putStr `` jjj''
            <- A =:= false

\end{lstlisting}
\end{frame}

\section{Substitution}

The higher order unification algorithm described is only defined on the pattern form provided in the previous section.
As the pattern fragment is a restriction on $\beta\eta$-long-normal form, it is necessary to provide a normalizing
substitution that preserves $\beta\eta$-long-normal form.  The presentation here
revolves around hereditary substitution, a notion first described by Pfenning et al. \citep{pfenning1991logic} for
a calculus with only $\beta$-reduction.

Other presentations are possible, such as traditional substitution followed by ``Normalization by Evaluation''
for $\beta\eta$-conversions\citep{abel2010towards}. 
While this is a proven total method in a typed setting, it's mechanics
are complex and not particularly enlightening.  
Keller et al. \citep{keller2010normalization}
extended hereditary substitution to $\eta$-expansion, but only in the simply typed case. 
The presentation here extends this version into $CC$.

\subsection{Untyped Substitution}

\begin{definition}
\textbf{(Hereditary Substitution)} 

\[ \begin{array}[3]{llr}
[ S / x ] x := S
&
[S / x] y := y
&
[S / x] P\;T := \m{H}([S/x] P, [S/x]T)
\end{array} \]

\[
[S / x] \lambda v : T . P := \lambda v' : [S/x]T . [S/x][v'/v]P
\;
\text{  where $v'$ is new.}
\]

\[ 
\m{H}(\lambda v : T . P , A) := [A/v] P
\]

\[ \begin{array}[2]{lr}
\m{H}(P A_1 , A_2) := P\;A_1\;A_2
&
\m{H}(V , A) := V\; A
\end{array} \]

\label{def:shered}
\end{definition}

It is important to note the alpha conversion in the $\lambda$ case, as alpha conversion will be lost on 
some terms when implicits are added.

A hereditary substitution is not necessarily terminating as is shown by
substitution replicating the $\omega$-combinator.

\[
[(\lambda x : T . x\; x) / x ] ( x \; (\lambda x : T. x\; x) )
\]

This is not defined, as it expands to the well known $\m{H}(\lambda x . x \; x , \lambda x . x \; x)$.

If the pattern and substitution are well typed terms in the Calculus of Constructions, by 
strong normalization, this version of hereditary substitution is total.


\begin{theorem} Substitution Theorem:

If $\Gamma,x : T \vdash A : T' : \m{prop}$  and $\Gamma \vdash S : T : \m{prop}$ then
$ \Gamma \vdash [S/x]_o A : [S/x]_o T' : \m{prop}$
\end{theorem}

By consistency, $[S/x]_o A $ can be normalized to strong head normal form.  Thus, the 
hereditary substitution $[S/x] A$ is defined.



\subsection{Typed Substitution}

Performing substitutions that maintain long $\beta\eta$-normal 
form is important to ensuring decidability of unification.  
Unfortunately this is not possible without some type information, 
as arbitrary $\eta$ expansion has no stop condition.
Keller \citep{keller2010normalization} gave a heredetary substitution algorithm
that results in canonical forms for simply typed lambda calculus. 
\ref{def:tyhered} solves this by performing a typed substitution under a context, 
generating type information.

\begin{theorem}
\textbf{(Reduction Decomposition)}
If $\Gamma \vdash A \Rightarrow_{\beta\eta*} B$ then there exists some $C$ such that
$A \Rightarrow_{\beta*} C$ and $\Gamma \vdash C \Rightarrow_{\eta*} B$

\label{thm:betaeta}
\end{theorem}

The property in \ref{thm:betaeta} stating that any reduction can be 
shown equivalent to first a series of $\beta$ reductions and then a series of
$\eta$ expansions forms the basis of the following algorithm.

For this section it is convenient to include $\Pi$ in the spine form: 
\[
N ::= P 
\orr \lambda V : N . N 
\orr \Pi V : N . N 
\]


\begin{definition}
\textbf{(Typed Hereditary Substitution)}

\begin{align}
[S / x : A]^n_{\Gamma } P &:= \m{E}([S / x : A]^p_{\Gamma } P)
\\
[S / x : A]^n_{\Gamma } (\lambda y : B . N) &:= \lambda y: [S / x ] B. [S / x : A]^n_{\Gamma, y : B} N
\\
\m{E}(M \uparrow P) &:= M
\\
\m{E}(N \downarrow A) &:= \eta^{-1}_A (N)
\\
\eta^{-1}_{\Pi x : A . B}(N) &:= \lambda z : A . \eta^{-1}_B(N \; \eta^{-1}_A(z))
 \text{ where $z$ is fresh}
\\
\eta^{-1}_{P}(N) &:= N
\\
[ N / x : A]^p_{\Gamma} x &:= N \uparrow A
\\
[S / x : A]^p_{\Gamma} y &:= y \downarrow \Gamma(x)
\\
[S / x : A]^p_{\Gamma } P\;N &:= 
\m{H}_{\Gamma} ( [S/x : A]^p_{\Gamma} P , [S/x : A]^n_{\Gamma} N) 
\\
\m{H}_{\Gamma} ((\lambda v : A_1 . N) \uparrow \Pi v' : A_1 . A_2 , P) 
&:= [P/v : A_1]^n_{\Gamma} N \uparrow [P/v']^n A_2
\\
\m{H}_{\Gamma}(P \downarrow \Pi y : B_1 . B_2 , N) &:= P\; N \downarrow [N/y]^nB_2
\end{align}

\label{def:tyhered}
\end{definition} 

In these last two rules, when initializing the dependent product, the substitution must be neither typed
nor $\eta$-expanding as this leads to an unnecessary circularity.  Types need not be in $\eta$-long form
when used for $\eta$-expanding in substitution, since they are not unified against anything.

Also, in the hereditary part of the recurrence, 
it is possible to omit cases for improperly formatted terms.  $A \downarrow T$ has the invariant that
$A$ must be already in a canonical form within the context it is used since it existed previously in the
equation.  Similarly, $A \uparrow T$ has the invariant that $A$ must be locally in a canonical form.
This permits significant reduction of steps.

In general, it is provable that for any PTS that is strongly normalizing for $\beta$ reduction and 
$\eta$ expansion, this algorithm will terminate and substitution will be defined.  
This is a direct consequence of \ref{thm:betaeta}.  
Since every step of the algorithm applies either an $\eta$ expansion or 
a $\beta$ reduction, the algorithm must halt when substituting for well typed terms.

The useful property of this hereditary substitution is spelled out in \ref{thm:hed-sound}

\begin{theorem}
\textbf{(Soundness of $\eta$ Expansion)}
If  $\Gamma \vdash F : A$ 
and $F$ is in $\beta$-normal form then
$\eta^{-1}_{A}(F)$ is in $\eta$-long form
\label{thm:hed-long}
\end{theorem}

The proof of this theorem is trivial since all that occurs is the complete $\eta$ expansion 
of every possible term. 

Note, in the following theorems \ref{thm:hed-sound}\ref{thm:hed-app1}\ref{thm:hed-app2} the $\Gamma \vdash N \; \m{Norm}$ means that $N $ is in $\beta$-normal 
$\eta$-long form (ignoring types).

\begin{theorem}
\textbf{(Soundness of Heredetary Substitution)}
If  $\Gamma \vdash S : A$ 
and $\Gamma \vdash S \; \m{Norm}$ 
and $\Gamma \vdash N \; \m{Norm}$ 
and $x$ is free for $S$ in $N$
and $[S / x : A]^n N$ is defined, 
then $\Gamma \vdash ([S / x : A]_\Gamma^n N) \; \m{Norm}$ 
\label{thm:hed-sound}
\end{theorem}
\begin{theorem}
\textbf{(Soundness of Heredetary Application-1)}
If  $\Gamma \vdash P : \Pi v' : A_1 . A_2$ 
and $\Gamma \vdash N : A_1$ 
and $\Gamma \vdash P \; \m{Norm}$ 
and $\Gamma \vdash N \; \m{Norm}$ 
then $E(H_\Gamma(P \downarrow \Pi v' : A_1 . A_2 , N)) \; \m{Norm}$ 
\label{thm:hed-app1}
\end{theorem}
\begin{theorem}
\textbf{(Soundness of Heredetary Application-2)}
If  $\Gamma \vdash (\lambda v : A_1 . N) : \Pi v' : A_1 . A_2$ 
and $\Gamma \vdash P : A_1$ 
and $\Gamma \vdash P \; \m{Norm}$ 
and $\Gamma \vdash N \; \m{Norm}$ 
and $v$ is free for $P$ in $N$
then $\Gamma \vdash E(H_\Gamma((\lambda v : A_1 . N) \uparrow \Pi v' : A_1 . A_2 , P)) \; \m{Norm}$ 
\label{thm:hed-app2}
\end{theorem}

The proof for these theorems is by mutual induction:  
for substitution on the structure of $[S/x : A]_\Gamma^n N = E([S/x : A]_\Gamma^p N)$, 
for application-1 on the structure of $E(H_\Gamma(P \downarrow \Pi v' : A_1 . A_2 , N))$,
and for application-2 on the structure of $E(H_\Gamma((\lambda v : A_1 . N) \uparrow \Pi v' : A_1 . A_2 , P))$.

\setcounter{tcases}{0}
\begin{tcases}
\textbf{Substitution} Suppose $N$ is of the form $(\lambda y : B . N')$.
\end{tcases}

\[
[S/x : A]_{\Gamma}^n (\lambda y : B . N') 
= \lambda y : [S / x] B . [S / x : A]^n_{\Gamma, y : B} N'
\]

And thus by the induction hypothesis, 
$\Gamma \vdash [S / x : A]^n_{\Gamma, y : B} N' \; \m{Norm}$
and thus
$\Gamma \vdash  \lambda y : [S / x] B . [S/x : A]_{\Gamma}^n N' \; \m{Norm}$

\begin{tcases}
\textbf{Substitution} Suppose $N$ is of the form $P \; N'$.
\end{tcases}

\[
[S/x : A]_{\Gamma}^n N 
= E( H_\Gamma([S/x : A]_{\Gamma}^n P, [S/x : A]_{\Gamma}^n N') )
\]

And thus by the induction hypothesis, 
$\Gamma \vdash [S/x : A]_{\Gamma}^n P \; \m{Norm}$
and
$\Gamma \vdash  [S/x : A]_{\Gamma}^n N' \; \m{Norm}$

Thus, $E( H_\Gamma([S/x : A]_{\Gamma}^n P, [S/x : A]_{\Gamma}^n N') )$ by 
mutual induction using \ref{thm:hed-app1}.

\begin{tcases}
\textbf{Substitution} Suppose $N$ is of the form $x$.
\end{tcases}

\[
[S/x : A]_{\Gamma}^n x 
= E( S \uparrow A)
= S
\]

and by argument, $\Gamma \vdash S\; \m{Norm}$ already.

\begin{tcases}
\textbf{Substitution} Suppose $N$ is of the form $y$.
\end{tcases}

\[
[S/x : A]_{\Gamma}^n y 
= E( y \downarrow \Gamma(y))
= \eta_{\Gamma(y)}^{-1}(y)
\]

which by theorem \ref{thm:hed-long} should be in $\beta$-normal, $\eta$-long form.

\begin{tcases}
\textbf{Application-1} 
\end{tcases}

\[
E(H_\Gamma(P \downarrow \Pi v' : A_1 . A_2 , N))
= 
E(P N \downarrow [N/v'] A_2)
=
\eta_{[N/v'] A_2}^{-1}(P N)
\]

Since $\Gamma \vdash N \; \m{Norm}$ and $\Gamma \vdash P \; \m{Norm}$, 

Thus, $P N$ is in $\beta$-normal form, and so 
$\Gamma \vdash \eta_{[N/v'] A_2}^{-1}(P N) \; \m{Norm}$

\begin{tcases}
\textbf{Application-2} 
\end{tcases}

\[
E(H_\Gamma((\lambda v : A_1 . N) \uparrow \Pi v' : A_1 . A_2 , P))
=
E( [ P / v : A_1 ]_\Gamma^n N \uparrow [ P / v' ]^n A_2)
=
[P / v : A_1 ]_\Gamma^n N
\]

Since $\Gamma \vdash (\lambda v : A_1 . N) : \Pi v' : A_1 . A_2$, we know that
$(\lambda v : A_1 . N) P$ must normalize and thus $[P/ v : A_1 ]_\Gamma^n N$ is defined.
Thus, since $\Gamma \vdash P \; \m{Norm}$ and $\Gamma \vdash N \; \m{Norm}$ and 
$\Gamma \vdash P : A_1$, we know that
$\Gamma \vdash [P/ v : A_1 ]_\Gamma^n N \; \m{Norm}$

Thus, 
$\Gamma \vdash E(H_\Gamma((\lambda v : A_1 . N) \uparrow \Pi v' : A_1 . A_2 , P)) \; \m{Norm}$

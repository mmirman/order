\section{Higher Order Unification}

Unification lies at the heart of the semantics of the Caledon language.

Checking for the reducability of two full lambda terms has long been known to be only semidecidable.  
The matter becomes even more complicated when checking for the equality of terms with variables bound
by both existential and universal quantifiers.  Research from the past thirty years has constrained
the problem to the a decidable subset where unification is decidable. 



\subsection{Unification Terms}

\begin{definition}
Unification Terms:

\[
U ::= U \wedge U 
 \orr \forall V : T . U
 \orr \exists V : T . U 
 \orr U \doteq U
 \orr \top
\]
\label{def:hou:syn}
\end{definition}

When $\doteq$ is taken to mean $\equiv_{\beta\eta\alpha*}$, the unification problem is to determine 
whether a statement $U$ is ``true'' in the common sense, and provide a proof of the truth of the statement. 

Unification problems of the form 
$\forall x : T_1 . \exists y : T_2 . U $ can be converted to the form
$\exists y : \Pi x : T_1 . T_2 . \forall x : T_1 . [y\; x / y ]U $ 
in the process known as raising. Unification
statements can always quantified over unused variables: $U \implies Q x : T . U$.  

Thus, statements can always be converted to the form
\[
\exists y_1 \cdots y_n . \forall x_1 \cdots x_k . S_1 \doteq V_1 \wedge \cdots S_r \doteq V_r
\]




\subsection{Unification Term Meaning}

We can provide an provability relation of a unification formula
based on the obvious logic.

\begin{definition}
$\Gamma \Vdash F $ can be interpreted as $\Gamma$ implies $F$ 
is provable.

\[ \begin{array}{lcr}
\infer[\m{equiv}]{
\Gamma \Vdash M \doteq N
}{
\Gamma \vdash M : A
&
M \equiv_{\beta\eta\alpha*} N
&
\Gamma \vdash N : A
}
&
\infer[\m{true}]{
\Gamma \Vdash \top
}{}
&
\infer[\m{conj}]{
\Gamma \Vdash F \wedge G
}{
\Gamma \Vdash F
&
\Gamma \Vdash G
}
\end{array} \]

\[ \begin{array}{lr}
\infer[\m{exists}]{
\Gamma \Vdash \exists x : A . F
}{
\Gamma \Vdash [M/x] F
&
\Gamma \vdash M : A
}
&
\infer[\m{forall}]{
\Gamma \Vdash \forall x : A . F
}{
\Gamma, x : A \Vdash F
}
\end{array} \]

\label{def:hou:prf}
\end{definition}

While a truly superb logic programming language might 
be able to convert this very declarative 
specification into a runnable program, 
the essentially nondeterministic rule for existential
quantification in a unification formula prevents an 
obvious deterministic algorithm from being extracted.


\subsection{Higher Order Unification for CC}

\newcommand{\UnifiesTo}{\;\longrightarrow\;}


I now present an algorithm, similar to that presented in 
\citep{pfenning1991logic} for unification in the 
Calculus of Constructions.  Because we have already 
presented typed hereditary substitution with $\eta$-expansion, 
the presentation here will not add much 
but for types in the substitutions.  

$F \UnifiesTo F'$ shall mean that $F$ can be transformed to $F'$
without modifying the provability. 
An equation $F[G]$ will stand as notation for highlighting $G$
under the formulae context $F$.  
As an example, if we were to examine the formula 
$\forall x . \forall n . \exists y . ( y \doteq x \wedge \forall z . \exists r . [ x z \doteq r] )$
but were only interested in the last portion, we might instead write it as
$\forall x . F[\forall z . \exists r . [ x z \doteq r]]$
Again, $\phi$ shall be an injective partial permutation. 

Furthermore, rather than explicitly writing down the result of unification, 
we shall use $\exists x. F \UnifiesTo \exists x . [ L / x] F$ 
to stand for $\exists x. F \UnifiesTo \exists x . x \doteq L \wedge [ L / x] F$


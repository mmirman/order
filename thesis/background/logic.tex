\section{Logic Programming}

Logic programming languages such as Prolog were originally designed as part of the ``AI
program'', in much the same way Lisp was. Automated reasoning’s natural goal was
to be able to arbitrarily prove theorems. A logic programming language was a 
set of axioms and a predicate. If the predicate could be proven through those axioms,
the automated theorem prover would halt.  These proof search procedures were
then constrained into useful programming semantics. When performed in a backtracking
manner, the proof-search process represented a formulation of procedural code
with powerful pattern matching. The Caledon language is a higher-order backtracking
logic programming language in the style of Elf \citep{pfenning1991logic}. In this section I present some basic
intuition for logic programming, rather than explaining it technically, and demonstrate the 
descriptive power of the system implemented in Caledon.

\FloatBarrier
\subsection{Basics}
We begin by defining addition on unary numbers in Caledon shown in shown in \ref{code:add}.

\begin{figure}[H]
\begin{lstlisting}
defn add : nat -> nat -> nat -> prop
   | addZ = add zero A A
   | addS = add (succ A) B (succ C) 
             <- add A B C
\end{lstlisting}
\caption{Addition in Caledon}
\label{code:add}
\end{figure}

One might notice that this definition is incredibly similar to its Haskell counterpart shown in \ref{code:hask}.

\begin{figure}[H]
\begin{lstlisting}
add :: nat -> nat -> nat
add Zero a = a
add (Succ a) b = Succ c
   where c = add a b
\end{lstlisting}
\caption{Addition in Haskell}
\label{code:hask}
\end{figure}

We can read the logic programming definition as we would read the functional definition with pattern
matching, knowing that an intelligent compiler would be able to convert the first into the second.  
Search allows one to define essentially nondeterministic programs. A common use for
logic programming has been to search for solutions to combinatorial games such as tic-tac-toe, 
without the programmer worrying about the order of the search. As this tends
to produce inefficient code, this use style is discouraged. Rather, a more procedural view of logic programming is encouraged where pattern match and search is performed
in the order it appears.

\begin{figure}[H]
\begin{lstlisting}
defn p : T_1 -> ... -> T_r -> prop
  >| n1 = p T_1 ... T_r <- p_1,1 ... <- p_1,k_1
...
  >| nN = p T_1 ... T_r <- p_n,1 ... <- p_n,k_n

query prg = p t1 ... tr
\end{lstlisting}
\caption{Format of a Caledon Logic Program}
\label{code:format}
\end{figure}

In this view, a program of the form \ref{code:format}
should be considered a program which first attempts to prove using axiom n1 by matching prg with ``p T1 ... Tr'' and then
attempting to prove $p_{1,1}$ and so on.


\FloatBarrier
\subsection{Higher Order Programming}

Fortunately, higher order functions need not be restricted to patterns.  
Definitions provide even more ways to generalize code. 

A great example is the function application operator from Haskell.  
We can define this in Caledon as shown in \ref{code:macros}

\begin{figure}[H]
\begin{lstlisting}
fixity right 0 @
defn @ : (At -> Bt) -> At -> Bt
  as ?\ At Bt . \ f : At -> Bt . \ a : At . f a

\end{lstlisting}
\caption{Definitions for expressive syntax}
\label{code:macros}
\end{figure}

In many cases, allowing these definitions allows for significant simplification of syntax.
The reader familiar with languages like Elf, Haskell, and Agda might notice the 
implicit abstraction of the type variables $At$ and $Bt$ in the type of $@$ in \ref{code:macros}. 
The rest of this paper is concerned with formalizing these implicit abstractions and letting them have
as much power as possible. For example, one might make these abstractions explicit by
instead declaring at the beginning \ref{code:expHask}

\begin{figure}[H]
\begin{lstlisting}
infixr 0 @
(@) :: forall At Bt . (At -> Bt) -> At -> Bt
(@) = \ f . \ a . f a
\end{lstlisting}
\caption{Explicit Haskell style abstractions}
\label{code:expHask}
\end{figure}

However, in a dependently typed language, every function type is a dependent product
(forall).  This makes it necessary to provide a new (explicit) implicit dependent product - $?\forall$ or $?\Pi$.


\begin{figure}[H]
\begin{lstlisting}
fixity right 0 @
defn @ : {At Bt:prop} (At -> Bt) -> At -> Bt
  as ?\ At Bt . \ f . \ a . f a

\end{lstlisting}
\caption{The (explicit) implicit equivalent of \ref{code:macros}}
\label{code:expimp}
\end{figure}

Haskell also has type classes. For example, the type of “show” can be seen in \ref{code:showty}.

\begin{figure}[H]
\begin{lstlisting}
show :: Show a => a -> String
\end{lstlisting}
\caption{The type of show}
\label{code:showty}
\end{figure}

In Caledon, these can be written similarly as in \ref{code:cshowty}

\begin{figure}[H]
\begin{lstlisting}
defn show : showC A => A -> string
defn show : {unused : showC A } A -> string
\end{lstlisting}
\caption{Equivalent types for show in Caledon}
\label{code:cshowty}
\end{figure}

However, since implicit arguments are a natural extension of the dependent type
system in Caledon, no restrictions are made on the number of arguments, or difficulty
of computing. Unfortunately, since computation is primarily accomplished by the logic
programming fragment of the language rather than the functional fragment of the language,
the correspondence between these programmable implicit arguments and type
classes is not one to one. It is possible to replicate virtually all of the functionality of
type classes in the implicit argument system, but the syntax required to do so can become
verbose. Rather than attempting to simulate type classes, more creative uses are
possible, such as computing the symbolic derivative of a type for use in a (albeit, slow
and unnecessary) generic zipper library, or writing programs that compile differently
with different types in different environments.




\section{Initial Examples}


In the previous section I gave an introduction to the notion of logic programming using
both the familiar language of Haskell and the new language of Caledon. In this section
I will build upon these ideas by introducing logic programming with polymorphism
through building a set of standard polymorphic type logic library.

There are a few ways of defining sums in Caledon.


\begin{figure}[H]
\begin{lstlisting}
defn and : type -> type -> type
   | pair = [A B : type] A -> B -> and A B

defn fst : and A B -> A -> type
   | fstImp = fst (pair Av Bv) Av

defn snd : and A B -> B -> type
   | sndImp = snd (pair Av Bv) Bv
\end{lstlisting}
\caption{Logical conjunction}
\label{code:lconj}
\end{figure}

In this first, simplest way (as seen in figure \ref{code:lconj}, we define a predicate for “and” and
predicates for construction and projection. This method has the advantage of doubling
as a form of sequential predicate.

\begin{figure}[H]
\begin{lstlisting}

query main = and (print ``hello ``) (print `` world!'')
\end{lstlisting}
\caption{Use of logical conjunction}
\label{code:lconjuse}
\end{figure}

In the figure \ref{code:lconjuse} the query will output ``hello world!''.

\begin{figure}[H]
\begin{lstlisting}

defn and : type -> type -> type
  as \ a : type . \ b : type . 
      [ c : type ] (a -> b -> c) -> c

defn pair : A -> B -> and A B
  as ?\ A B : type .
      \ a b .
      \ c : type .
      \ proj : A -> B -> c .
        proj a b

defn fst : and A B -> A
  as ?\ A B : type .
      \ pair : [c : type] (A -> B -> c) -> c .
        pair A (\ a b . a)

defn snd : and A B -> A
  as ?\ A B : type .
      \ pair : [c : type] (A -> B -> c) -> c .
        pair B (\ a b . b)

\end{lstlisting}
\caption{Church style conjunction}
\label{code:cconj}
\end{figure}

In the case of figure \ref{code:cconj}, we do not add any axioms without their proofs.   
In this example we also introduce the dependent product written in the form $[ a : t_1 ] t_2$.

This case mimics the version usually seen in the Calculus of Constructions and has the advantage of
the projections being functions rather than predicates.


\begin{figure}[H]
\begin{lstlisting}

defn churchList : type −> type
  as \A : type . [ lst : type] lst −> (A −> lst −> lst ) −> lst

defn consCL : [ B : type ] B -> churchList B -> churchList B
  as \ C : type .
     \ V : C .
     \ cl : churchList C .
     \ lst : type .
     \ nil : lst .
     \ cons : C -> lst -> lst .
     cons V (cl lst nil cons)

defn nilCL : [ B : type ] churchList B
  as \ C : type .
     \ lst : type .
     \ nil : lst .
     \ cons : C -> lst -> lst .
       nil

defn mapCL : { A B } ( A -> B) -> churchList A -> churchList B
  as ?\ A B : type .
      \ F : A -> B.
      \ cl : churchList A .
      \ lst : type .
      \ nil : lst .
      \ cons : B -> lst -> lst .
        cl lst nil (\ v . cons (F v))

defn foldrCL : { A B } ( A -> B -> A) -> A -> churchList B -> A 
  as ?\ A B : type . 
      \ F : A -> B -> A .
      \ bc : A .
      \ cl : churchList B .
        cl A bc (\ v : B . \ c : A . F c v)

\end{lstlisting}
\caption{Church style list}
\label{code:clist}
\end{figure}

In the Church form of a list, folds and maps are possible to implement as functions
rather than predicates. However, their implementation is verbose and doesn't permit
more complex functions.


\begin{figure}[H]
\begin{lstlisting}

defn list : type -> type
   | nil = list A
   | cons = A -> list A -> list A

defn concatList : list A -> list A -> list A -> type
   | concatListNil = [ L : list A ] concatList nil L L
   | concatListCons = 
          concatList (cons (V : T) A) B (cons V C) 
       <- concatList A B C

defn concatList : list A -> list A -> list A -> type
   | concatListNil = [ L : list A ] concatList nil L L
   | concatListCons = 
          concatList (cons (V : T) A) B (cons V C) 
       <- concatList A B C

defn mapList : (A -> B) -> list A -> list B -> type
   | mapListNil = [ F : A -> B] mapList F nil nil
   | mapListCons = [F : A -> B] 
            mapList F (cons V L) (cons (F V) L')
           <- mapList F L L'

defn pmapList : (A -> B -> type) -> list A -> list B -> type
   | pmapListNil = [ F : A -> B -> type] pmapList F nil nil
   | pmapListCons = [F : A -> B -> type] 
            pmapList F (cons V L) (cons V' L')
            <- F V V'
            <- mapList F L L'
\end{lstlisting}
\caption{Logic List}
\label{code:llist}
\end{figure}

The logic programming version can be seen in figure \ref{code:llist}. It is important to note that
we can now map a predicate over a list rather than just mapping a function over a list.
